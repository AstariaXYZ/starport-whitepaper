\documentclass[12pt]{article}

\usepackage[utf8]{inputenc}
\usepackage{multicol}
\usepackage{blindtext}
\usepackage{graphicx}
\usepackage[dvipsnames]{xcolor}
\graphicspath{ {./images/} }
%\pagecolor{Periwinkle}

\title{Astaria v1 Lending \& Intents}
\author{Andrew Redden \\ \texttt{andrew@astaria.xyz}
        \and
        Joseph Delong \\ \texttt{joe@astaria.xyz}
        \and
        Gregory Curtis \\ \texttt{greg@astaria.xyz}}
\date{\today}

\begin{document}

\maketitle

\begin{abstract}
Astaria v1 is a lending protocol built on the Starport\cite{starport} lending kernel. Astaria v1 is an over collateralized oracleless intents based fixed rate open term lending protocol.\\

Astaria v1 supports an arbitrary basket of ERC-721\cite{erc721}, ERC-1155\cite{erc1155}, ERC-20\cite{erc20} (min 1e6 max 1e18) as collateral assets, and an individual ERC-20 borrow asset.
\end{abstract}

\section{Introduction}
Astaria v1 was designed to allow borrowers to optimize their rate and loan-to-value (LTV) with reduced interactivity. Additionally the design reduces liveness requirements for lenders. 
\section{Intents}
Every origination in Astaria v1 is an intent. The intent is a timed linear inclining rate auction with amount bounds. Borrowers sign an EIP-712\cite{eip712} message off chain. The message is then used by the lender to fill the conditions of the auction.

The borrower signed message contains these key parameters, start time $\ t_0$, end time $\ t_{end}$, start rate $\ r_0$, end rate $\ r_{end}$, minimum amount $\ a_{min}$, and maximum amount $\ a_{max}$.

To calculate the current satisfying rate$\ r_c$ use the following equation:

$$
r_c=\frac{\Delta t(r_{end}-r_{0}) }{t_{end} - t_{0}} \\
$$


The end rate will satisfy the auction until the EIP-712 signature expires.\\

Calculate the satisfying amount$\ a$ using the following inequality:
$$
a_{min} \le a \le a_{max}
$$

The lender maintains optionality on the amount provided in the auction as a lender in an efficient market will lend to their maximum LTV on a risk-adjusted basis for the collateral at the calculated rate.

This novel approach allows the borrower to optimize on axis of amount and rate simultaneously. 

\section{Fixed Rates}
Once an intent is filled the loan is a fixed rate compound interest loan. The borrower repays the lender the principal with compound interest.

Interest is calculated using the continuously compounding exponential growth formula.

$$
A = A_0e^{rt}
$$

The benefit of this approach is gas efficiency, accuracy of interest calculations, and preventing a compounding penalty for refinancing.

\section{Open Term \& Recall}
All loans in Astaria v1 are open term, meaning loans have no predefined end date. Loans continue until they are recalled. Recall is the process of replacing a an original lender with a refinancing lender or impairing the loan and seizing the collateral.\\

There are no oracles in Astaria v1, instead recall acts as the mechanism for impairing or refinancing loans.

\section{Recall Intent}
If a lender wants to exit their position, the lender initiates a recall using an on chain transaction.\\

Recall opens a new intent for an inclining rate auction. The configuration of this auction is parameterized but the general case is the loan goes from 0-1000\% over 24 hours.\\

If no lender refinances (or the borrower has not repaid the loan) the loan in the recall period the loan is presumed defaulted also called a failed recall. The collateral is now retrievable by the lender through an on chain transaction.\\

Borrowers and lenders can recall at anytime through the course of the loan.\\

The following parameters control the conditions of a recall:
\itemize
\item honeymoon$\ h$ is a duration after origination in which a loan cannot be recalled
\item recall window$\ w$ is the duration of the recall intent
\item maximum rate$\ \rho$ is the end rate for a recall

\subsection{Conditions}
A loan can only be recalled one time. A loan cannot be recalled until honeymoon period has been exceeded.

\subsection{Recallers}
A lender or borrower can recall at anytime except after a recall has been initiated.

\subsection{Recall Auction}

Once a loan is recalled, it begins a recall auction where the recaller and the recall start are stored. A recall auction is a linear inclining rate auction going from$\ \approx0$ to the maximum rate over the duration of the recall window.\\

During this period any user can refinance the recalled loan for a rate that is equal or below the current rate of the recall auction.\\

Which can be calculated using the following equation:

$$
 r_c = \frac{\rho \Delta t}{w}
$$

If the recall auction fails to achieve a refinance by the end of the recall window the recall auction is deemed defaulted and the lender can take possession of the collateral.\\

If the loan is refinanced the lender is repaid by the refinancing lender. The collateral is reassigned to the refinanced loan.

\begin{thebibliography}{9}
\bibitem{erc20}
Fabian Vogelsteller, Vitalik Buterin. 
\textit{ERC-20: Token Standard}. 
Ethereum Improvement Proposals, 2015.\\
https://eips.ethereum.org/EIPS/eip-20

\bibitem{erc721}
 William Entriken, Dieter Shirley, Jacob Evans, Nastassia Sachs. 
\textit{ERC-721: Non-Fungible Token Standard}. 
Ethereum Improvement Proposals, 2018.\\
https://eips.ethereum.org/EIPS/eip-721

\bibitem{erc1155}
Witek Radomski, Andrew Cooke, Philippe Castonguay, James Therien, Eric Binet, Ronan Sandford. 
\textit{ERC-1155: Multi Token Standard}. 
Ethereum Improvement Proposals, 2018.\\
https://eips.ethereum.org/EIPS/eip-1155

\bibitem{eip712}
Remco Bloemen, Leonid Logvinov, Jacob Evans.
\textit{EIP-712: Typed structured data hashing and signing} 
Ethereum Improvement Proposals, 2017.\\
https://eips.ethereum.org/EIPS/eip-712

\bibitem{starport}
Andrew Redden, Joseph Delong, Gregory Curtis.
\textit{Starport: Lending Kernel} 
Astaria Github, 2024.\\
https://github.com/AstariaXYZ/starport-whitepaper/blob/master/starport-whitepaper.pdf

% Add more references as needed
\end{thebibliography}
\end{document}

